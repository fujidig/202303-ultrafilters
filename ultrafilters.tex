\documentclass[uplatex]{jsarticle}
\usepackage[utf8]{inputenc}

\usepackage{amssymb}
\usepackage{amsmath}
\usepackage{amsthm}
\usepackage{framed}
\usepackage{braket}
\usepackage{bm}
\usepackage{mathrsfs}
\usepackage{accents}
\usepackage{tocloft}
\usepackage[dvipdfmx]{graphicx}
\usepackage{tikz}
\usepackage{url}
\usepackage{color}
\usepackage{xifthen}
\usepackage{xcolor}
\usepackage{framed}
\usepackage{mathtools}
\usepackage[explicit]{titlesec}
\usepackage{mdframed}
\usepackage{geometry}
\geometry{left=30mm,right=30mm,top=20mm,bottom=20mm}
\usepackage{enumerate}
\usepackage[dvipdfmx]{hyperref}
\usepackage{pxjahyper}
\renewcommand{\baselinestretch}{1.1}

\usetikzlibrary{positioning}
\usetikzlibrary{calc}
\usetikzlibrary{decorations.pathreplacing}
\usetikzlibrary{cd}


\renewcommand{\labelenumi}{(\arabic{enumi})}

\newcommand{\scrN}{\mathcal{N}}
\newcommand{\scrI}{\mathcal{I}}
\newcommand{\scrC}{\mathcal{C}}
\newcommand{\scrJ}{\mathcal{J}}
\newcommand{\N}{\mathbb{N}}
\newcommand{\Z}{\mathbb{Z}}
\renewcommand{\P}{\mathbb{P}}
\newcommand{\B}{\mathbb{B}}
\newcommand{\Q}{\mathbb{Q}}
\newcommand{\R}{\mathbb{R}}
\newcommand{\C}{\mathbb{C}}
\newcommand{\range}{\operatorname{ran}}
\newcommand{\dom}{\operatorname{dom}}
\newcommand{\append}{{}^\frown}
\newcommand{\boldsig}{\boldsymbol{\Sigma}}
\newcommand{\boldpi}{\boldsymbol{\Pi}}
\newcommand{\bolddelta}{\boldsymbol{\Delta}}
\newcommand{\Ordinals}{\mathrm{On}}
\newcommand\forces{\Vdash}
\newcommand\notforces{\nVdash}
\newcommand{\cl}{\operatorname{cl}}
\newcommand{\intr}{\operatorname{int}}
\newcommand{\ro}{\operatorname{ro}}
\newcommand{\rank}{\operatorname{rank}}
\newcommand{\frakt}{\mathfrak{t}}
\newcommand{\s}{\mathfrak{s}}
\newcommand{\frakb}{\mathfrak{b}}
\newcommand{\frakd}{\mathfrak{d}}
\newcommand{\frakc}{\mathfrak{c}}
\newcommand{\frakg}{\mathfrak{g}}
\newcommand{\fraku}{\mathfrak{u}}
\newcommand{\Pow}{\mathcal{P}}
\newcommand{\non}{\operatorname{non}}
\newcommand{\cov}{\operatorname{cov}}
\newcommand{\add}{\operatorname{add}}
\newcommand{\cof}{\operatorname{cof}}
\newcommand{\Cof}{\mathbf{Cof}}
\newcommand{\Cov}{\mathbf{Cov}}
\newcommand{\D}{\mathbf{D}}
\newcommand{\Lc}{\mathbf{Lc}}
\newcommand{\nul}{\mathsf{null}}
\newcommand{\meager}{\mathsf{meager}}
\newcommand{\id}{\mathrm{id}}
\newcommand{\diam}{\mathrm{diam}}
\newcommand{\height}{\mathrm{ht}}
\newcommand{\pow}{\mathrm{pow}}
\newcommand{\GTle}{\preceq_\mathrm{GT}}
\newcommand{\Map}[2]{\operatorname{Map}(#1, #2)}
\newcommand{\omegaupomega}{\omega^{\uparrow \omega}}
\newcommand{\twototheltomega}{2^{<\omega}}
\newcommand{\cf}{\operatorname{cf}}
\newcommand{\LangL}{\mathcal{L}}
\newcommand{\Add}{\operatorname{Add}}
\newcommand{\Seq}{\operatorname{Seq}}
\newcommand{\stem}{\operatorname{stem}}
\newcommand{\suc}{\operatorname{succ}}
\newcommand{\Lev}{\operatorname{Lev}}
\newcommand{\AND}{\mathbin{\&}}
\newcommand{\OR}{\text{ or }}
\newcommand{\restrict}{\upharpoonright}
\newcommand{\Lim}{\mathrm{Lim}}
\newcommand{\Limone}{\mathrm{Lim}_{\omega_1}}
\newcommand{\ZFC}{\mathsf{ZFC}}
\newcommand{\CH}{\mathsf{CH}}
\newcommand{\subsetic}{\subseteq_{\mathrm{ic}}}
\newcommand{\crit}{\operatorname{crit}}
\newcommand{\Ult}{\operatorname{Ult}}
\newcommand{\ext}{\operatorname{ext}}
\newcommand{\statone}{\mathsf{stat}_{\omega_1}}
\newcommand{\Coll}{\operatorname{Coll}}
\DeclareMathOperator*{\diagintr}{\triangle}
\DeclareMathOperator*{\diaguni}{\bigtriangledown}

\newcommand{\seq}[1]{{\langle#1\rangle}}
\DeclarePairedDelimiter\abs{\lvert}{\rvert}
\DeclarePairedDelimiter\floor{\lfloor}{\rfloor}
\DeclarePairedDelimiter\ceil{\lceil}{\rceil}

\renewcommand\emptyset{\varnothing}
\renewcommand\subset{\subseteq}
\renewcommand{\setminus}{\smallsetminus}

\newcommand{\needtocheck}[1][]{%
	\ifthenelse{\equal{#1}{}}{%
		\textcolor{blue}{[NeedToCheck]}%
	}{%
		\textcolor{blue}{[NeedToCheck: #1]}%
	}%
}

\newcommand{\todo}[1][]{%
	\ifthenelse{\equal{#1}{}}{%
		\textcolor{red}{[TODO]}%
	}{%
		\textcolor{red}{[TODO: #1]}%
	}%
}


\theoremstyle{definition}
\newtheorem{thm}{定理}[section]
\newtheorem*{thm*}{定理}
\newtheorem{defi}[thm]{定義}
\newtheorem*{defi*}{定義}
\newtheorem{lem}[thm]{補題}
\newtheorem*{lem*}{補題}
\newtheorem{fact}[thm]{事実}
\newtheorem*{fact*}{事実}
\newtheorem{prop}[thm]{命題}
\newtheorem*{prop*}{命題}
\newtheorem{exm}[thm]{例}
\newtheorem*{exm*}{例}
\newtheorem{rmk}[thm]{注意}
\newtheorem*{rmk*}{注意}
\newtheorem{cor}[thm]{系}
\newtheorem*{cor*}{系}
\newtheorem*{notation*}{記法}
\newtheorem{asm}[thm]{仮定}
\newtheorem{prob}[thm]{演習問題}
\newtheorem{conj}[thm]{予想}
\newtheorem{defiandlem}[thm]{定義と補題}
\renewcommand{\proofname}{証明}

\newenvironment{claim}[1]{\par\noindent\underline{主張 #1:}\space}{}
\newenvironment{claimproof}[1]{\par\noindent$\because$) \space#1}{\hfill //}

\usepackage[backend=biber,style=alphabetic,sorting=nty,doi=false,isbn=false,url=false,eprint=true]{biblatex}
\addbibresource{ultrafilters.bib}
\renewbibmacro{in:}{}


\usepackage{titling}
\renewcommand\maketitlehooka{
	\vspace{-1.5cm}
	\noindent\vrule height 2.5pt width \textwidth
}
\pretitle{
	\begin{center}
		\Huge\bfseries
		\vspace{-0.5cm}
	}
	\posttitle{
	\end{center}
}
\preauthor{
	\begin{flushright}
		\large
		\vspace{-0.5cm}
	}
	\postauthor{
	\end{flushright}
}
\predate{
	\begin{flushright}
		\large
		\vspace{-0.5cm}
	}
	\postdate{
	\end{flushright}
	\vspace{-0.5cm}\noindent\vrule height 2.5pt width \textwidth
}

\title{超フィルターと実数の集合論}
\date{2023年3月*日 作成}
\author{でぃぐ}

\begin{document}
	
	\maketitle
	
	\begin{abstract}
	\end{abstract}
	
	\tableofcontents
	
	\section{基数不変量の定義}
	
	\begin{defi}
		\begin{enumerate}
			\item $\omega^\omega$上の関係$\le^*$を$x \le^* y$とは、ある$n \in \omega$があって全ての$m \ge n$で$x(m) \le y(m)$であることと定める。
			\item $\Pow(\omega)$上の関係$\subset^*$を$x \subset^* y$とは、$x \setminus y$が有限集合となることと定める。
			\item $F \subset \omega^\omega$が\textbf{unbounded family}であるとは$(\forall x \in \omega^\omega)(\exists y \in F)(y \not \le^* x)$を満たすことを言う。$\frakb = \min \{ \abs{F} : F \subset \omega^\omega \text{はunbounded family} \}$と定め、\textbf{bounding number}という。
			\item $F \subset \omega^\omega$が\textbf{dominating family}であるとは$(\forall x \in \omega^\omega)(\exists y \in F)(x \le^* y)$を満たすことを言う。$\frakd = \min \{ \abs{F} : F \subset \omega^\omega \text{はdominating family} \}$と定め、\textbf{dominating number}という。
			\item $\mathcal{F} \subset \Pow(\omega)$が$\omega$上の\textbf{超フィルターのベース}であるとは、$\{ A \subset \omega : (\exists B \in \mathcal{F})(B \subset^* A) \}$が$\omega$上の非単項超フィルターとなることを言う。
			$\fraku = \min \{ \abs{\mathcal{F}} : \mathcal{F}\text{は$\omega$上の超フィルターのベース} \}$と定め、\textbf{ultrafilter number}という。
			\item $\mathcal{G} \subset [\omega]^\omega$が\textbf{groupwise dense}であるとは、$\mathcal{G}$がほとんど部分集合の関係で閉じていて、かつ、任意の区間分割$\seq{I_n : n \in \omega}$について、ある$A \in [\omega]^\omega$があって、$\bigcup_{n \in A} I_n \in \mathcal{G}$となることを言う。$\frakg = \min \{ \abs{\mathcal{G}} : \mathcal{G} \subset [\omega]^\omega \text{はgroupwise dense} \}$と定め、\textbf{groupwise density}という。
		\end{enumerate}
	\end{defi}
	
	以下の図のような順序が知られている。ここで矢印$A \to B$は$A \le B$が$\ZFC$で証明できることを意味する。
	証明は全てBlassの記事 \cite{blass2010combinatorial}に載っている。
	
	\[
	\tikz{
		\node (aleph1) at (-2, -2) {$\aleph_1$};
		\node (b) at (0, 0) {$\frakb$};
		\node (d) at (2, 0) {$\frakd$};
		\node (covm) at (2, -2) {$\cov(\meager)$};
		\node (g) at (0, -2) {$\frakg$};
		\node (u) at (4, -1) {$\fraku$};
		\node (c) at (4, 1) {$\frakc$};
		
		\draw[->] (aleph1) -- (b);
		\draw[->] (aleph1) -- (g);
		\draw[->] (b) -- (d);
		\draw[->] (covm) -- (d);
		\draw[->] (g) -- (d);
		\draw[->] (covm) -- (u);
		\draw[->] (d) -- (c);
		\draw[->] (u) -- (c);
	}
	\]
	
	\section{痩せフィルター}
	
	関数$f$が有限対一とは、任意の1点集合の逆像が有限となることを言う。
	
	\begin{defi}
		$\mathcal{F} \subset \Pow(\omega)$と$f \colon \omega \to \omega$に対して、
		\[
		f(\mathcal{F}) = \{ X \subset \omega : f^{-1}(X) \in \mathcal{F} \}
		\]
		と定める。
	\end{defi}
	
	\begin{defi}
		フィルター$\mathcal{F}$が\textbf{痩せフィルター} (feeble filter)であるとは、ある有限対一の関数$f \colon \omega \to \omega$が存在して、$f(F)$がFréchetフィルターに一致することを言う。
	\end{defi}
	
	\begin{thm}
			$\frakb$個未満の集合で生成されるフィルターはすべて痩せフィルターである。なおかつ、この$\frakb$個未満というのは次の意味で最適:痩せてないフィルターで$\frakb$個の集合で生成されるものがある。
	\end{thm}
	\begin{proof}
		フィルター$\mathcal{F}$とそのベース$\mathcal{B}$で$\abs{\mathcal{B}} < \frakb$なものを考える。
		各$A \in \mathcal{B}$に対して、区間分割$\Pi_A$であって、そのどの区間も$A$の元を持つものを取る。
		$\Pi_A$たちの個数は$\frakb$個未満なので、ある一個の区間分割$\Pi'$が取れて、全ての $\Pi_A$ ($A \in \mathcal{B}$)を支配する。
		すると$\mathcal{F}$のどの元$A$についても$A$は$\Pi'$に属する区間の有限個を除いた全てと交わる。
		よって命題???より痩せフィルターである。
	\end{proof}

\nocite{*}
\printbibliography[title={参考文献}]


	
\end{document}
